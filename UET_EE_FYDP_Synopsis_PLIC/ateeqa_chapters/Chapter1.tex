\chapter{Introduction}
\label{Chapter1}
An Embedded System communicates with the outside world through its input/output devices. The computer can acquire information through input devices, and it can show information through output devices. Output devices also allow the computer to manipulate its environment. An embedded system differs from a standard computer system by having a close connection to the outside world. The difficulty is that, in the majority of cases, software runs considerably more quickly than hardware. For instance, it might take the software only 1 ${\mu}$s to ask the hardware to clear the LCD, but the hardware might take 1 ms to complete the command. During this time, software may carry out ten thousands of instructions. As a result, for an embedded system application to be successful, the synchronisation between the executing software and its external environment is necessary. \\

This synchronization is somehow managed by sending an interrupt.  An interrupt is an unexpected event that occurs during normal working of a system. It is the automatic transfer of software execution in response to a hardware event that is asynchronous with the current software execution.\\

In computer architecture mostly external devices sends interrupt.  External devices are comparatively slower than CPU. So if there is no interrupt CPU would waste a lot of time waiting for external devices to match its speed with that of CPU. This decreases the efficiency of CPU. Hence, interrupt is required to eliminate these limitations.\\

The processor services the interrupts. In RISC-V interrupts are classified into timer, software and external interrupts. The external interrupts are also called as global interrupts. Timer and software interrupts are handled by a Core Local Interrupt (CLINT). External interrupts are handled by the \textbf{PLIC}.\\

\textbf{Platform level interrupt controller} is shortly known as PLIC. The PLIC connects global interrupt sources, which are usually I/O devices, to interrupt targets, which are usually hart contexts.
The I/O devices need attention so it will interrupt one of the core. For example human types character keyboard will send interrupt to any of the core through PLIC and as a result that core will talk directly to the keyboard. It will run an interrupt handler. A trap will occur and then core will make interrupt handler to run and ask keyboard what character was typed. When the operation is finished the core will return to its original operation and resume its work.\\

The PLIC contains multiple interrupt gateways, one per interrupt source, together with a PLIC core that performs interrupt prioritization and routing. Global interrupts are sent from their source to an interrupt gateway that processes the interrupt signal from each source and sends a single interrupt request to the PLIC core, which latches these in the core interrupt pending bits (IP). Each interrupt source is assigned a separate priority. The PLIC core contains a matrix of interrupt enable (IE) bits to select the interrupts that are enabled for each target. The PLIC core forwards an interrupt notification to one or more targets if the targets have any pending interrupts enabled, and the priority of the pending interrupts exceeds a per-target threshold. When the target takes the external interrupt, it sends an interrupt claim request to retrieve the identifier of the highest priority global interrupt source pending for that target from the PLIC core, which then clears the corresponding interrupt source pending bit. After the target has serviced the interrupt, it sends the associated interrupt gateway an interrupt completion message and the interrupt gateway can now forward another interrupt request for the same source to the PLIC.