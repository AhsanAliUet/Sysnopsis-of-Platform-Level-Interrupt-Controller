% Chapter 3

\chapter{Literature Review}
\label{Chapter3}

Interrupts are Asynchronous events generated by a external source via hardware. In RISC-V interrupts are classified into timer,
software and external interrupts. The external interrupts are also called as global interrupts. Timer and software interrupts are handled by a Core Local Interrupt. External interrupts are handled by the PLIC.

\section{Working}

The PLIC connects the global interrupt sources to the interrupt target i.e., core. The PLIC consists of the ”PLIC core” and the ”Interrupt gateways”. There are multiple interrupt gateways, one per interrupt source. Global interrupts are sent from their source to one of the interrupt gateway. The interrupt gateway processes the arriving interrupt signal from each source and sends a single interrupt request to the PLIC core. The PLIC core contains a set of interrupt enable bits to enable individual interrupt sources in the PLIC. The PLIC core contains pending interrupt bits to signal that an interrupt is waiting to be processed. Also, PLIC core performs interrupt prioritization. Each interrupt source is assigned a separate priority. The PLIC core latches the interrupt request into the Interrupt Pending bits. Whenever, the priority of the pending interrupt exceeds a per-target threshold, the PLIC core forwards an interrupt notification to the interrupt target. The PLIC Claim register holds the highest priority interrupt waiting to be processed. On interrupt completion the interrupt Gateways can now send another interrupt request to the PLIC.

\subsection{Operation Parameters}

Register blocks that perform PLIC operation parameters are Interrupt Priorities registers for the selection of interrupt priority for each interrupt source. The second one is Interrupt Pending Bits register for the interrupt pending status of each interrupt source. The third one is  Interrupt Enables register to perform the enablement of interrupt source of each context. The fourth one is  Priority Thresholds register to select the interrupt priority threshold of each context. The fifth one is Interrupt Claim register to acquire interrupt source ID of each context. And finally the Interrupt Completion register to send interrupt completion message to the associated gateway.

\subsection{Memory Register Map}
The base address of PLIC Memory Map is platform implemented and its width is 32-bit.
\begin{center}
\begin{tabular}{|c|c|c|} 
 \hline
Register Adress. & Data Width & Description \\ [0.5ex] 
 \hline
 0x0C000000 & 4 bytes & Source zero priority(base Addr) \\ [1ex]
 \hline
 0x0C000004 & 4 bytes & Source 1 priority \\ [1ex]
 \hline
.\\[0.2ex]
.\\[0.2ex]

\hline
 0x0C002003 & 8 bytes & Interrupt Enabled Source 24 to 27 \\ [1ex]
 \hline
 0x0C000000 & 4 bytes & Priority Threshold register \\ [1ex]
 \hline
 0x0C010010 & 4 bytes & Interrupt claim \\ [1ex]
\hline
\end{tabular}

\end{center}

\section{Interrupt Prioritues}
Interrupt priorities are small unsigned integers, with a platform-specific maximum number of supported levels. The priority value 0 is reserved to mean "never interrupt", and interrupt priority increases with increasing integer values. Each global interrupt source has an associated interrupt priority held in a memory-mapped register. Different interrupt sources need not support the same set of priority values. A valid implementation can hardwire all input priority levels. Interrupt source priority registers should be WARL fields to allow software to determine the number and position of read-write bits in each priority specification, if any. To simplify discovery of supported priority values, each priority register must support any combination of values in the bits that are variable within the register, i.e., if there are two variable bits in the register, all four combinations of values in those bits must operate as valid priority levels. The base address of Interrupt Source Priority block within PLIC Memory Map region is fixed at
0x000000.

 \begin{center}
 \begin{tabular}{|c|c|c|} 
 \hline
PLIC register block name & Register Block size in byte & Function \\ [0.5ex] 
 \hline
 Interrupt Source Priority & 1024*4 = 4096(0x1000) bytes & Interrupt Source Priority 0 to 1023  \\ [1ex]
 \hline
\end{tabular}

\end{center}

\subsection{Interrupt Pending Register}

The current status of the interrupts pending in the PLIC core can be read from the interrupt pending register. The interrupt pending register is a set of 2, 32 bit words. It can be seen as a array of 8 bytes. The pending bit of interrupt id 0 is stored in LSB of first pending register. The pending bit for interrupt ID N is stored in the N mod 8th bit of N/8th byte. The PLIC has 2 interrupt pending registers. Bit 0 of byte 0 represents the non-existent interrupt source 0 and is hardwired to zero. A pending bit in the PLIC core can be cleared by setting the associated enable bit then performing a claim as described in section. The content of the Interrupt pending register is
read-only.


\section{Interrupt Enablers and Priority Thresholds}
Each global interrupt can be enabled by setting the corresponding bit in the enables register. The enables registers are accessed as a contiguous array of 32-bit registers, packed the same way as the pending bits. Bit 0 of enable register 0 represents the non-existent interrupt ID 0 and is hardwired to 0. PLIC has 15872 Interrupt Enable blocks for the contexts.In PLIC a large number of potential IE bits might be hardwired to zero in cases where some interrupt sources can only be routed to a subset of targets. A larger number of bits might be wired to 1 for an embedded device with fixed interrupt routing. Interrupt priorities, thresholds, and hart-internal interrupt masking provide considerable flexibility in ignoring external interrupts even if a global interrupt source is always enabled.

PLIC provides context based threshold register for the settings of a interrupt priority threshold of each context. The threshold register is a WARL field. The PLIC will mask all PLIC interrupts of a priority less than or equal to threshold. For example, a`threshold` value of zero permits all interrupts with non-zero priority.

\subsection{Interrupt Completion}
When PLIC signals completes executing an interrupt handler by writing the interrupt ID it received from the claim to the claim register. The PLIC does not check whether the completion ID is the same as the last claim ID for that target. If the completion ID does not match an interrupt source that is currently enabled for the target, the completion is silently ignored. After a handler has completed service of an interrupt, the associated gateway must be sent an interrupt completion message, usually as a write to a non-idempotent memory mapped I/O control register. The gateway will only forward additional interrupts to the PLIC core after receiving the
completion message.







